\chapter{Giới thiệu}
Nhiều bài toán giao vận trong thực tế như giao hàng, dịch vụ vận chuyển... có thể được mô hình hóa bằng bài toán định tuyến xe (vehicle routing problems - VRP) [10]. VRP giải quyết vấn đề định tuyến một số xe với trọng tải giới hạn để phục vụ toàn bộ yêu cầu của khách hàng với chi phí nhỏ nhất, thường được đo bằng số xe nhân với tổng quãng đường di chuyển. Thường thì mỗi khách hàng có một khung thời gian khác nhau với thời điểm sớm nhất và muộn nhất họ có thể nhận hàng, điều này dẫn đến bài toán VRP với khung thời gian (time windows - VRP-TWs). Nói cách khác, xe phải đến chỗ khách hàng trong một khoảng thời gian nhất định trong bài toán VRP-TWs. Việc xe đến trước thời gian hẹn sớm nhất của khách hàng sẽ dẫn đến thời gian nhàn rỗi, ngược lại xe không được đến muộn hơn thời gian nhận hàng muộn nhất mà khách hàng yêu cầu. Hơn nữa, thời gian phục vụ thường tùy thuộc vào từng khách hàng. Một ví dụ thực tế cho VRP-TWs là dịch vụ giao báo, trong đó khách hàng thường sẽ nhận hàng vào một khung giờ (time windows) nhất định trong ngày. VRP-TWs đã được chứng minh là một bài toán NP-đầy đủ, yêu cầu thời gian mũ cho thuật toán tổng quát trong trường hợp xấu nhất. Trong thực tế, rất nhiều trường hợp của VRP-TWs với 100 khách hàng [1, 10] hoặc hơn [9] khó giải một cách tối ưu [1, 3, 6]. Trong 2 thập kỉ qua, VRP-TWs với bản chất thách thức và giá trị thực tiễn của nó đã liên tục được nghiên cứu và có nhiều thuật tóan mẹo cũng như thuật toán tìm kiếm [1, 3] để giải một cách hiệu quả với trí tuệ nhân tạo [3], lập trình ràng buộc [1] và nghiên cứu vận hành [3].


Trong số các phương pháp heuristics [1, 2] được đề xuất để giải bài toán định tuyến xe, có một vài đề xuất thú vị để khởi tạo tìm kiếm trong khi một số khác tập trung vào việc thúc đẩy tìm kiếm từ một trạng thái ban đầu một cách hiệu quả. Liên quan đến việc khởi tạo tìm kiếm, có 2 phương pháp hiệu quả đó là \textit{push-forward insertion heuristic} (PFIH) [6] và \textit{ virtual vehicle heuristic} (VVH) [2]. Các phương pháp này được đề xuất để tạo ra các trạng thái khởi tạo khả thi để dẫn tới kết quả tốt hơn. 